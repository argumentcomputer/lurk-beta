\documentclass[10pt, english]{article}
\usepackage[]{graphicx}
\usepackage[]{color}
\usepackage{alltt}
\usepackage[T1]{fontenc}
\usepackage[utf8]{inputenc}
\setcounter{secnumdepth}{3}
\setcounter{tocdepth}{3}
\setlength{\parskip}{\smallskipamount}
\usepackage{lipsum}
\usepackage{amsmath}
\usepackage{amsfonts}
\usepackage{amsthm}
\usepackage{mathabx}
\usepackage{tikz}
\usepackage[noend]{algpseudocode}
\usepackage[section]{algorithm}
\usepackage{algorithmicx}
\usepackage{xspace}
\usepackage{fancyvrb}
\usepackage{url}
\usepackage[titletoc]{appendix}

\newcommand{\redexp}{\mathrm{reduce\_expression}}
\newcommand{\redwithwit}{\mathrm{reduce\_with\_witness}}
\newcommand{\redcons}{\mathrm{reduce\_cons}}
\newcommand{\redsym}{\mathrm{reduce\_sym}}
\newcommand{\applycont}{\mathrm{apply\_continuation}}
\newcommand{\carcdr}{\mathrm{car\_cdr}}
\newcommand{\makethunk}{\mathrm{make\_thunk}}

\begin{document}

\begin{titlepage}
	\newcommand{\HRule}{\rule{\linewidth}{0.5mm}}

	\clearpage\thispagestyle{empty}
	\centering
	\vspace{1cm}

	\textsc{\Large Specification}\\[.5cm]
	\HRule\\[.5cm]
	{\Huge Zero-Knowledge circuit \par}
	\vspace{.5cm}
	{\Large for continuation-passing interpreter \par}
	\vspace{.5cm}
	\HRule \\[.5cm]


\includegraphics[width=0.4\textwidth,scale=1]{logo.png}\\[.1cm]
	{\large \today}\\[2cm]
\vfill
\end{titlepage}


\begin{abstract}
  This document describes how Lurk circuits are constructed. It is a \textbf{work in progress}.
  Right now it contains only a short overview of the specification. A complete description will be provided in the near future.
\end{abstract}

\tableofcontents

\newpage

\section{Introduction}

Lurk is a functional programming language based on Lisp. An important concept in its design is Continuation Passing Style (CPS)~\footnote{A good source of information on CPS is the book ``Essentials of Programming Languages''~\cite{FWbook}}, where the control flow of programs can be managed through the use of \textit{continuations}, which represent the rest of the computation. This technique allows us to divide the program into small parts. As a result, we can build a small circuit for each part. We will summarize the main concepts involved in the design of the language, such that the reader can understand how zero-knowledge proofs~\cite{groth16, nova, halo2} are constructed to provide privacy to Lurk's programs.

\section{Lurk}

In this section, we provide a summary of Lurk's main elements. An \textit{expression} represents a computation involving literals, variables, operations and procedures. Variables are handled by an \textit{environment}, which is responsible for binding variables and values. We also use \textit{continuations} to indicate what must done to finish the computation.

The system's I/O is formed by an expression, an environment, and a continuation. These 3 elements are essential to comprehend Lurk. Each element is represented as a pointer, which is implemented using hash functions. In particular, we use Poseidon~\cite{poseidon} to instantiate our pointers.

The environment has a list of bindings, which correspond to a mapping between variables and values at a certain point in time. Therefore, it has local bindings, meaning that the mapping is valid only for a specific evaluation of an expression. On the other hand, the global state is represented by the \textit{store}, which behaves as a memory of the system. The store is global, while the environment is local.

\subsection{Overview}

\begin{itemize}
  \item[-] \textbf{t, nil:} are self-evaluating and represent true and false, respectively.
  \item[-] \textbf{if:} has the format \verb|(if <test> <consequent> <alternate>)| and represents a conditional expression. It \textbf{must} receive all 3 parameters, where the \verb|<test>| is an expression used to select the result; the \verb|<consequent>| is selected if the \verb|<test>| evaluates to non-\verb|nil|; and \verb|<alternate>| is selected otherwise. Unlike other programming languages, the \verb|<alternate>| expression is mandatory.
  \item[-] \textbf{lambda:} has the format \verb|(lambda <formals> <body>)| and represents a procedure. The environment when the lambda expression is evaluated is used as a \textbf{closure}, which is extended with \verb|<formals>|, a list of variables. The unique expression in the \verb|<body>| is evaluated and returned as the result of the lambda expression.
  \item[-] \textbf{let:} has the format \verb|(let <bindings> <body>)| and represents an assignment expression, where \verb|<bindings>| represents a list of pairs in the form \verb|(<variable>, <init>)|; and \verb|<body>| is a unique expression.
  \item[-] \textbf{letrec:} has the same format as \verb|<let>| expressions, following the same rules, but also allowing recursion.
  \item[-] \textbf{quote:} has the format \verb|(quote <datum>)| or \verb|(' <datum>)| and evaluates to \verb|<datum>|.
  \item[-] \textbf{atom:} has the format \verb|(atom <e>)|, and it evaluates to \verb|t| if \verb|<e>| is not a list, and evaluates to \verb|nil| otherwise.
  \item[-] \textbf{cons, car, cdr:} The expression \verb|(cons <a> <d>)| produces a pair whose \verb|car| is \verb|<a>| and \verb|cdr| if \verb|<d>|.
  \item[-] \textbf{arithmetic operations:} has the format \verb|(<op> <e1> <e2>)|, where \verb|<op>| corresponds to an arithmetic operation ($+, -, \star, \slash$). \verb|<e1>| is evaluated before \verb|<e2>| and the operation is carried out in the finite field that is used in the subjacent zero-knowledge backend.
  \item[-] \textbf{equality:} has the format \verb|(<op> <e1> <e2>)|, where \verb|<op>| can be either \verb|=| or \verb|eq|. The equality symbol \verb|=| is used to compare expressions whose result is a number (finite field elements), while the symbol \verb|eq| is used to compare pointers.
  \item[-] \textbf{emit:} has the format \verb|(emit <e>)| and is used to return the result of the evaluation of \verb|<e>| as a public value, which can be used to define the instance of the zero-knowledge statement.
  \item[-] \textbf{begin:} has the format \verb|(begin <e> ...)|. The sequence of expressions is evaluated from left to right and the last result is returned.
  \item[-] \textbf{current env:} returns the current environment represented as an association list.
  \item[-] \textbf{eval:} has the format \verb|(eval <exp>)| or \verb|(eval <exp> <env>)|. The evaluation of \verb|<exp>| is used as an expression in the environment obtained from the evaluation of \verb|<env>|. If no \verb|<env>| is provided, an empty environment is used.
\end{itemize}

\subsubsection{Fibonacci example}

Here is an example code snippet that implements the Fibonacci's sequence:

\begin{verbatim}
>
(letrec ((next (lambda (a b n target)
        (if (eq n target)
            a
                (next b
                (+ a b)
                (+ 1 n)
                    target))))
            (fib (next 0 1 0)))
        (fib 10))
[521 iterations] => 55
\end{verbatim}

\section{Circuit}

Here we describe the construction of Lurk's circuit.

\subsection{Overview}

A Lurk program is split into a sequence of \textit{reduction steps}, also called iterations. In the Fibonacci example above, we have 521 iterations, and each one is mapped into a \textit{frame}. We group a set of frames into a MultiFrame object.

We construct a CircuitFrame for each frame, and a Circuit is a sequence of CircuitFrames, where the output of a previous one is linked to the input of the next. The circuit mimics the evaluation of Lurk expressions.

In \verb|eval.rs| it is possible to see the implementation of evaluation of Lurk's expressions. An important function is called $\redwithwit()$, which is responsible for the computation of reduction steps with its witnesses. Since a reduction step is exactly what we want to prove in zero-knowledge, we provide an implementation that carries out the same computation, but in the circuit. This implementation is accomplished by $\redexp()$.

We call \textit{global symbols} the set of Lurk symbols that are pre-computed, such that we can easily compare with symbols found during the evaluation of expressions.

To reduce an expression, we require a substantial number of conditions that depend on comparisons among symbols. Those conditions primarily involve allocation of variables and pointers in the circuit, Boolean logic using those elements, and conditionals. This functionality is implemented in method $\redsym()$. The Boolean logic and equality tests are executed against global symbols and other auxiliary variables created along the process in order to make decisions about the control flow of Lurk programs. We can update the environment and the store accordingly.

One of the most important building blocks in a functional language like Lisp and Lurk is \verb|cons|. It is used to concatenate \verb|car| and \verb|cdr|. Hence, whenever we want to break big expressions into smallers ones, we can use a combination of those elements. In function $\redcons()$ we deal with each different situation where those elements are necessary. Namely, to reduce an expression into smaller pieces, we allocate in the circuit some auxiliary variables for later utilization. In particular, $\carcdr()$ gadget is used as a building block. Since this computation is required for different situations, depending on the type of expression we are handling, we include a clause in a multicase gadget for each situation. We can easily select the desired result using this approach. The multicase selection is based on the \verb|head| of an expression, which we get using \verb|car|. Finally, we return the result of the multicase.

Another important function is $\applycont()$. In order to finish a reduction step, we must calculate the output of the frame, which is what we are going to explain now. Each iteration has a continuation tag, and for each one we need to compute the next expression, environment, continuation, and thunk. Therefore, we have to constrain the system to prove we are computing the correct elements, and we have to allocate pointers to use them later. This task is executed in 2 stages:

\begin{itemize}
  \item[-] Some continuations require the calculation of new pointers, while others don't. For those which need new pointers, since the implementation of pointers requires a hash computation and because hashes are expensive in the circuit, we use a multicase to select the appropriate hash preimage. Then we can compute the hashes just once. This allows us to avoid computing unnecessary hashes.
	\item[-] We then use another multicase to select the continuation results.
\end{itemize}

\subsection{Gadgets}

In order to construct the circuit we use some gadgets and auxiliary functions as building blocks. Full detail such as the number of constraints of each component and their implementation description will be provided soon.

\begin{itemize}
	\item[-] Macros:
	\begin{itemize}
		\item \textbf{Boolean:} it is used to handle bit operations like conjunctions, disjunctions and negations.
    \item \textbf{Equality:} it allows to compare allocated variables.
		\item \textbf{Conditional:} it permits to construct \verb|if-then-else| expressions.
		\item \textbf{Pick:} we use it for ternary operators.
		\item \textbf{Implication:} this is used for constraints in the form: if $a$ is true, then $b$ is true, where $a$ and $b$ are expressions that evaluate to Boolean values.
	\end{itemize}
	\item[-] Constraints:
	\begin{itemize}
    \item \textbf{Arithmetic operations:} We can use it to constrain arithmetic operations ($+, -, \star, \slash$) in the subjacent finite field.
	  \item[-] \textbf{Pointers:} we have that pointers are formed by a \emph{tag}, which allows to indentify the type of the pointer, and a \emph{hash} that links the pointer to its content, which is given by the hash preimage.
    \item[-] \textbf{Data:} we have functions to allocate different types of data by using pointers. Later we can access the data using dictionaries.
    \item[-] \textbf{Multicase:} this component is used to select results based on certain selection tags. It is basically a set of \verb|cases| that shares the same set of selection tags. A multicase whose size is equal to 1 is the same as a regular \verb|case|.
  \end{itemize}
\end{itemize}

\section{Final remarks}

In this document we presented a preliminary sketch of Lurk's circuit, wherein Lurk's programs can be proved using zero-knowledge.

\section{References}

\bibliographystyle{plain}

\bibliography{refs}

\end{document}
